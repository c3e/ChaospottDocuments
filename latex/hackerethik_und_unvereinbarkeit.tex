% 
% Hab das mal fertig gemacht. Bandie
%
% Printer settings:
%   - 2 pages per paper
%   - Duplex: Short edge (flip)
%
\documentclass[a5paper]{scrartcl}
\pagenumbering{gobble}
\usepackage[ngerman]{babel}
\usepackage[T1]{fontenc}
\usepackage{lmodern}
\usepackage{graphicx}
\usepackage{transparent}
\usepackage{draftwatermark}
\SetWatermarkText{{\transparent{0.08}\includegraphics[width=15cm]{../pdf/chaosknoten.pdf}}}
\title{\textbf{Hackerethik}}
\subtitle{Die ethischen Grundsätze des Hackens – Motivation und Grenzen}
\date{}
\begin{document}
\maketitle

\begin{itemize}
 \item Der Zugang zu Computern und allem, was einem zeigen kann, wie diese Welt funktioniert, sollte unbegrenzt und vollständig sein.
 \item Alle Informationen müssen frei sein.
 \item Misstraue Autoritäten – fördere Dezentralisierung.
 \item Beurteile einen Hacker nach dem, was er tut, und nicht nach üblichen Kriterien wie Aussehen, Alter, Herkunft, Spezies, Geschlecht oder gesellschaftliche Stellung.
 \item Man kann mit einem Computer Kunst und Schönheit schaffen.
 \item Computer können dein Leben zum Besseren verändern.
 \item Mülle nicht in den Daten anderer Leute.
 \item Öffentliche Daten nützen, private Daten schützen.
\end{itemize}


Die Hackerethik befindet sich – genauso wie die übrige Welt – insofern in ständiger Weiterentwicklung und Diskussion. Dabei dürfen natürlich alle mitdenken, die sich grundsätzlich mit dieser Hackerethik anfreunden können. Bis dahin stehen die o. g. Regeln als Diskussionsgrundlage und Orientierung.

\newpage

\maketitle

\begin{itemize}
 \item Der Zugang zu Computern und allem, was einem zeigen kann, wie diese Welt funktioniert, sollte unbegrenzt und vollständig sein.
 \item Alle Informationen müssen frei sein.
 \item Misstraue Autoritäten – fördere Dezentralisierung.
 \item Beurteile einen Hacker nach dem, was er tut, und nicht nach üblichen Kriterien wie Aussehen, Alter, Herkunft, Spezies, Geschlecht oder gesellschaftliche Stellung.
 \item Man kann mit einem Computer Kunst und Schönheit schaffen.
 \item Computer können dein Leben zum Besseren verändern.
 \item Mülle nicht in den Daten anderer Leute.
 \item Öffentliche Daten nützen, private Daten schützen.
\end{itemize}


Die Hackerethik befindet sich – genauso wie die übrige Welt – insofern in ständiger Weiterentwicklung und Diskussion. Dabei dürfen natürlich alle mitdenken, die sich grundsätzlich mit dieser Hackerethik anfreunden können. Bis dahin stehen die o. g. Regeln als Diskussionsgrundlage und Orientierung.


%% Teil Unvereinbarkeitserklärung
\title{\textbf{Unvereinbarkeitserklärung}}
\subtitle{Farbe bekennen gegen Rechts}

\newpage

\maketitle

Wir sind eine galaktische Gemeinschaft von Lebewesen, unabhängig von Alter, Geschlecht und Abstammung sowie gesellschaftlicher Stellung, offen für alle mit neuen Ideen. Wer jedoch mit Ideen von Rassismus, Ausgrenzung und damit verbundener struktureller und körperlicher Gewalt auf uns zukommt, hat sich vom Dialog verabschiedet und ist jenseits der Akzeptanzgrenze. 
\newline

\textbf{Wer es darauf anlegt, das Zusammenleben in dieser Gesellschaft zu zerstören und auf eine alternative Gesellschaft hinarbeitet, deren Grundsätze auf Chauvinismus und Nationalismus beruht, arbeitet gegen die moralischen Grundsätze, die uns als Club verbinden.\newline}

Der Chaos Computer Club erklärt das Vertreten von Rassismus und von der Verharmlosung der historischen und aktuellen faschistischen Gewalt für \underline{unvereinbar} mit einer Mitgliedschaft.

\newpage

\maketitle

Wir sind eine galaktische Gemeinschaft von Lebewesen, unabhängig von Alter, Geschlecht und Abstammung sowie gesellschaftlicher Stellung, offen für alle mit neuen Ideen. Wer jedoch mit Ideen von Rassismus, Ausgrenzung und damit verbundener struktureller und körperlicher Gewalt auf uns zukommt, hat sich vom Dialog verabschiedet und ist jenseits der Akzeptanzgrenze. 
\newline

\textbf{Wer es darauf anlegt, das Zusammenleben in dieser Gesellschaft zu zerstören und auf eine alternative Gesellschaft hinarbeitet, deren Grundsätze auf Chauvinismus und Nationalismus beruht, arbeitet gegen die moralischen Grundsätze, die uns als Club verbinden.\newline}

Der Chaos Computer Club erklärt das Vertreten von Rassismus und von der Verharmlosung der historischen und aktuellen faschistischen Gewalt für \underline{unvereinbar} mit einer Mitgliedschaft.

\end{document}

\documentclass[10pt,a4paper]{scrartcl}
\usepackage[utf8]{inputenc}
\usepackage[T1]{fontenc}
\usepackage[ngerman]{babel}
\usepackage[hidelinks]{hyperref}
\usepackage[top=30mm, bottom=45mm, left=35mm, right=35mm]{geometry}
\usepackage{enumitem}
\renewcommand{\familydefault}{\sfdefault}
\renewcommand{\thesection}{§\arabic{section}}

\begin{document}
	\noindent
	{\huge\bfseries Datenschutzerklärung gemäß DSGVO Art.~13\\
		für die Mitgliederverwaltung des foobar~e.V.}\\[0.5cm]
	Stand: 8.2.2022
\vspace{1cm}

\noindent Wir erheben auf dem Mitgliedsantrag persönliche Daten.
Gemäß Datenschutzgrundverordnung~(DSGVO) Artikel~13 geben wir dazu die
folgende Datenschutzerklärung ab:


\section*{(1a) Name und Anschrift des Verantwortlichen}

\subsection*{Verantwortlicher}

foobar e.V.\\
Sibyllastr. 9\\
45136 Essen\\
Deutschland\\[2ex]
E-Mail: info@die-foobar.de


\subsection*{Vertreter}

Der Vorstand:\\[2ex]
C. Roschow\\
S. Surminski\\
J. Stuber


\section*{(1b) Datenschutzbeauftragter}

Keiner, da weniger als 10 Personen mit der Verarbeitung
personenbezogener Daten beschäftigt sind (BDSG §4f Satz 4).


\section*{(1c)}

\subsection*{Zwecke der Verarbeitung}

Wir verarbeiten personenbezogene Daten unserer Mitglieder grundsätzlich
nur, soweit dies zur Vereinsführung erforderlich ist.
Regelmäßig sind dies z.B. die Einladung zu Versammlungen,
die Prüfung ob Mitgliedsbeiträge bezahlt wurden,
sowie die Ausstellung von Zuwendungsbescheinigungen.
%
Die Emailadresse kann bei Einwilligung des Mitglieds für die Einladung
zu Versammlungen verwendet werden.
%
Die Emailadresse wird zur sonstigen Kommunikation mit dem Mitglied
verwendet. Insbesondere wird sie zur internen Mailingliste
hinzugefügt, damit das Mitglied vereinsinterne Informationen erhält.
Gegebenenfalls ist es möglich, sich dort wieder abzumelden.


\subsection*{Rechtsgrundlage der Verarbeitung}

Rechtsgrundlage für die Verarbeitung ist DSGVO Art.~6~(1b), da die
Vereinsmitgliedschaft rechtlich gesehen ein Vertrag zwischen Mitglied
und Verein ist.

\section*{(2a) Erforderliche Daten}

Name und Postanschrift sind erforderlich zur Identifikation, für die
Benachrichtigung des Mitglieds über Versammlungen und für sonstige
formale Vereinsangelegenheiten wie zum Beispiel Satzungsänderungen.
%
Die Informationen zum Beitragshöhe und -turnus sind erforderlich
für die Prüfung, ob Mitgliedsbeiträge bezahlt wurden.


\section*{(2b) Freiwillige Daten}

Die Angabe einer E-Mail-Adresse ist nicht unbedingt erforderlich.
Für eine direkte Kommunikation innerhalb des Vereins ist es aber ratsam eine solche Adresse anzugeben.
Es kann angegeben werden, dass ausschließlich E-Mail-Kommunikation gewünscht wird.
In diesem Fall findet jegliche Kommunikation per E-Mail statt.
%
Die GPG-Schlüssel-Id ist nicht erforderlich, kann aber freiwillig angegeben werden.
Diese wird dazu verwendet, die Kommunikation via E-Mail für das Mitglied zu verschlüsseln.
%
Wir verarbeiten diese Daten auf Grundlage von DSGVO Art.~6~(1a)


\section*{(2d) Rechte}

Es bestehen Rechte auf Auskunft (siehe Art.~15),
Berichtigung (Art.~16),
Löschung (Art.~17),
Einschränkung der Verarbeitung von Daten (Art.~18)
sowie der Übertragung von Daten (Art.~20).

Außerdem besteht das Recht auf Widerruf gegen die Verarbeitung der freiwillig angegebenen Email-Adresse (Art.~7).

Es besteht kein Recht auf Widerspruch gegen die Verarbeitung erforderlicher Daten,
da DSGVO Art.~6~(1e und 1f) nicht einschlägig sind.

\section*{(2e) Beschwerderecht}

Es besteht das Recht, eine Beschwerde bei der zuständigen Aufsichtsbehörde einzureichen.
Das Beschwerderecht kann bei einer Aufsichtsbehörde in dem Mitgliedstaat
des gewöhnlichen Aufenthaltsorts,
des Arbeitsplatzes oder
des Orts des mutmaßlichen Verstoßes geltend gemacht werden.

\end{document}
